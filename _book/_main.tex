% Options for packages loaded elsewhere
\PassOptionsToPackage{unicode}{hyperref}
\PassOptionsToPackage{hyphens}{url}
%
\documentclass[
]{book}
\title{The Research Data Management Workbook}
\author{Kristin Briney}
\date{2023-08-14}

\usepackage{amsmath,amssymb}
\usepackage{lmodern}
\usepackage{iftex}
\ifPDFTeX
  \usepackage[T1]{fontenc}
  \usepackage[utf8]{inputenc}
  \usepackage{textcomp} % provide euro and other symbols
\else % if luatex or xetex
  \usepackage{unicode-math}
  \defaultfontfeatures{Scale=MatchLowercase}
  \defaultfontfeatures[\rmfamily]{Ligatures=TeX,Scale=1}
\fi
% Use upquote if available, for straight quotes in verbatim environments
\IfFileExists{upquote.sty}{\usepackage{upquote}}{}
\IfFileExists{microtype.sty}{% use microtype if available
  \usepackage[]{microtype}
  \UseMicrotypeSet[protrusion]{basicmath} % disable protrusion for tt fonts
}{}
\makeatletter
\@ifundefined{KOMAClassName}{% if non-KOMA class
  \IfFileExists{parskip.sty}{%
    \usepackage{parskip}
  }{% else
    \setlength{\parindent}{0pt}
    \setlength{\parskip}{6pt plus 2pt minus 1pt}}
}{% if KOMA class
  \KOMAoptions{parskip=half}}
\makeatother
\usepackage{xcolor}
\IfFileExists{xurl.sty}{\usepackage{xurl}}{} % add URL line breaks if available
\IfFileExists{bookmark.sty}{\usepackage{bookmark}}{\usepackage{hyperref}}
\hypersetup{
  pdftitle={The Research Data Management Workbook},
  pdfauthor={Kristin Briney},
  hidelinks,
  pdfcreator={LaTeX via pandoc}}
\urlstyle{same} % disable monospaced font for URLs
\usepackage{longtable,booktabs,array}
\usepackage{calc} % for calculating minipage widths
% Correct order of tables after \paragraph or \subparagraph
\usepackage{etoolbox}
\makeatletter
\patchcmd\longtable{\par}{\if@noskipsec\mbox{}\fi\par}{}{}
\makeatother
% Allow footnotes in longtable head/foot
\IfFileExists{footnotehyper.sty}{\usepackage{footnotehyper}}{\usepackage{footnote}}
\makesavenoteenv{longtable}
\usepackage{graphicx}
\makeatletter
\def\maxwidth{\ifdim\Gin@nat@width>\linewidth\linewidth\else\Gin@nat@width\fi}
\def\maxheight{\ifdim\Gin@nat@height>\textheight\textheight\else\Gin@nat@height\fi}
\makeatother
% Scale images if necessary, so that they will not overflow the page
% margins by default, and it is still possible to overwrite the defaults
% using explicit options in \includegraphics[width, height, ...]{}
\setkeys{Gin}{width=\maxwidth,height=\maxheight,keepaspectratio}
% Set default figure placement to htbp
\makeatletter
\def\fps@figure{htbp}
\makeatother
\setlength{\emergencystretch}{3em} % prevent overfull lines
\providecommand{\tightlist}{%
  \setlength{\itemsep}{0pt}\setlength{\parskip}{0pt}}
\setcounter{secnumdepth}{5}
\usepackage{booktabs}
\ifLuaTeX
  \usepackage{selnolig}  % disable illegal ligatures
\fi
\usepackage[]{natbib}
\bibliographystyle{plainnat}

\begin{document}
\maketitle

{
\setcounter{tocdepth}{1}
\tableofcontents
}
\hypertarget{about-this-book}{%
\chapter*{About this Book}\label{about-this-book}}
\addcontentsline{toc}{chapter}{About this Book}

\hypertarget{description}{%
\section*{Description}\label{description}}
\addcontentsline{toc}{section}{Description}

The Research Data Management Workbook is made up of a collection of exercises for researchers to improve their data management. The workbook contains exercises across the data lifecycle, though the range of activities is not comprehensive. Instead, exercises focus on discrete practices within data management that are structured and can be reproduced by any researcher.

The book is divided into chapters by phases of the data lifecycle, with one or more exercises in each chapter. Every exercise comes with a description of its value within data management, instructions on how to do the exercise, original source of the exercise, and the exercise itself.

The workbook is intended as a supplement to existing data management education. If you would like to learn more about the principles of data management, please see the article ``Foundational Practices of Research Data Management'' by Briney, Coates, and Goben \citep{briney_foundational_2020} or read the book ``Data Management for Researchers'' \citep{briney_data_2015}.

\hypertarget{license}{%
\section*{License}\label{license}}
\addcontentsline{toc}{section}{License}

This book is available under a Creative Commons Attribution-NonCommercial (CC BY-NC) 4.0 International license.

\hypertarget{the-author}{%
\section*{The Author}\label{the-author}}
\addcontentsline{toc}{section}{The Author}

\begin{figure}
\centering
\includegraphics{images/00_KristinBriney.jpg}
\caption{Headshot of author, Kristin Briney}
\end{figure}

Kristin Briney is the Biology \& Biological Engineering Librarian at the California Institute of Technology and author of the books ``Data Management for Researchers'' \citep{briney_data_2015} and, with Becky Yoose, ``Managing Data for Patron Privacy'' \citep{briney_managing_2022}. She has a PhD in chemistry and an MLIS, both from the University of Wisconsin-Madison. Her research focuses on research data management, institutional data policy, and patron privacy vis-a-vis library data handling. Kristin is an advocate for the adoption of the international date standard ISO 8601 (YYYY-MM-DD) and likes to spend her free time making data visualizations out of yarn and fabric.

\hypertarget{introduction}{%
\chapter{Introduction}\label{introduction}}

\hypertarget{documentation}{%
\chapter{Documentation}\label{documentation}}

\hypertarget{lab-notebook}{%
\section{Evaluate a Laboratory Notebook}\label{lab-notebook}}

\textbf{\emph{Description:}} \emph{The laboratory or research notebook is a fundamental documentation method for many researchers. But for how ubiquitous the lab notebook is, documentation can sometimes be lacking. The ideal laboratory notebook allows someone with similar training as you to be able to follow everything you did in your research. This exercise prompts you to review an old entry within your laboratory notebook to evaluate if your documentation is sufficient for reproducing your work.}

\textbf{\emph{Instructions:}} \emph{You will need a laboratory notebook entry from 6-12 months ago to do this exercise. Once you have the entry, read through it to try to understand what you did on that day. Answer the exercise questions to evaluate the entry and identify any note keeping improvements to make.}

\begin{center}\rule{0.5\linewidth}{0.5pt}\end{center}

\textbf{Date of lab notebook entry being evaluated:} \_\_\_\_\_\_\_\_\_\_\_\_\_\_\_\_\_\_\_\_

\textbf{Read the entry and summarize the work you did on that date:}

~

~

~

\textbf{How easy was it to understand what you did from your notes? Could you reproduce your work solely from the information in your notes?}

~

~

~

\textbf{What worked well with your note keeping?}

~

~

~

\textbf{What would you improve about your note keeping?}

~

~

~

\textbf{List one change you plan to make to take better research notes:}

~

~

~

\hypertarget{readme-txt}{%
\section{Write a Project-Level README.txt}\label{readme-txt}}

\textbf{\emph{Description:}} \emph{Data files living on a computer often need extra documentation for someone to understand what research they correspond to. In particular, it is useful to record the most basic project information and store it in the top-level folder of each research project. This can be done with a README.txt. The name, ``README'', indicates that the file conveys important information and the file type, TXT, can be opened by many different software programs, making the content maximally accessible. This exercise walks you through the key information needed in a project-level README.txt file. The same information can also be recorded at the front of a physical laboratory notebook.}

\textbf{\emph{Instructions:}} \emph{Pick a research project and answer the following questions. Copy all of the text into a TXT file and save it with the name ``README.txt''. Store this file in the top-level of the project folder on your computer, alongside the project files.}

\textbf{\emph{Source:}} \emph{This exercise was adapted from the ``Project Close Out Checklist'' \citep{briney_project_2020}.}

\begin{center}\rule{0.5\linewidth}{0.5pt}\end{center}

\textbf{What is the title of the project?}

~

~

~

\textbf{What is the project description?}

~

~

~

\textbf{What is the time period the project was done over?}

~

~

~

\textbf{Who worked on the project?}

~

~

~

\textbf{Where are the files are stored?}

~

~

~

\textbf{How are files organized? Are any naming conventions used and, if so, what are they (see \protect\hyperlink{file-organization-and-naming}{Chapter 3})?}

~

~

~

\hypertarget{data-dictionary}{%
\section{Create a Data Dictionary}\label{data-dictionary}}

\textbf{\emph{Description:}} \emph{Ideally, a spreadsheet is formatted with a row of variable names at the top, followed by rows of data going down. This makes easy for data to be used in any data analysis software (interoperability is a good thing) but makes it impossible to document a spreadsheet within the file itself. For this reason, it's useful to create a data dictionary to describe the spreadsheet so that others can interpret the data. This exercise walks you through the major information you should record for each variable in the spreadsheet, adding up to a complete dictionary to accompany the spreadsheet file.}

\textbf{\emph{Instructions:}} \emph{Fill out the question in each row for each variable in the spreadsheet; note that you will likely have more variables than columns in this table. Copy this information into a text document and save it next to the spreadsheet. It is useful to save the data dictionary with the same root name as its data file by appending ``\_dictionary'' on the end of the file name; for example, the data dictionary for the file ``myData.xlsx'' would be ``myData\_dictionary.txt''.}

\textbf{\emph{Source:}} \emph{This exercise was adapted from the ``Leveling Up Data Management'' \citep{briney_leveling_2023}.}

\begin{center}\rule{0.5\linewidth}{0.5pt}\end{center}

\begin{tabular}{l|l|l|l}
\hline
Question & Variable1 & Variable2 & Variable3\\
\hline
Variable name &  &  & \\
\hline
Variable description &  &  & \\
\hline
Variable units &  &  & \\
\hline
Relationship to other variables &  &  & \\
\hline
Variable coding values and meanings &  &  & \\
\hline
Known issues with the data &  &  & \\
\hline
Anything else to know about the data? &  &  & \\
\hline
\end{tabular}

\hypertarget{file-organization-and-naming}{%
\chapter{File Organization and Naming}\label{file-organization-and-naming}}

\hypertarget{file-organization}{%
\section{Set Up a File Organization System}\label{file-organization}}

\textbf{\emph{Description:}} **

\textbf{\emph{Instructions:}} **

\begin{center}\rule{0.5\linewidth}{0.5pt}\end{center}

\hypertarget{file-naming}{%
\section{Create a File Naming Convention}\label{file-naming}}

\textbf{\emph{Description:}} \emph{File naming conventions are a simple way to add order to your files and help to find them later. Rich and descriptive file names make it easier to search for files, understand at a glance what they contain, and tell related files apart.}

\begin{figure}
\centering
\includegraphics{images/03_FileNaming.jpg}
\caption{Screenshot of pdf's with consistent file names using the convention FirstAuthorLastName\_YEAR\_ShortTitle.pdf}
\end{figure}

\textbf{\emph{Instructions:}} \emph{This exercise guides researchers through the process of creating a file naming convention for a group of related files. Fill in each section for a group of related files following the instructions; an example for microscopy files is provided. This exercise may be redone as needed, as different groups of files require different naming conventions.}

\textbf{\emph{Source:}} \emph{This exercise is based on the ``File Naming Convention Worksheet'' \citep{briney_file_2020}.}

\begin{center}\rule{0.5\linewidth}{0.5pt}\end{center}

\textbf{1. What group of files will this naming convention cover?}

You can use different conventions for different file sets.

\emph{Example: This convention will apply to all of my microscopy files, from raw image through processed image.}

~

~

~

\textbf{2. What information (metadata) is important about these files and makes each file distinct?}

Ideally, pick three pieces of metadata; use no more than five. This metadata should be enough for you to visually scan the file names and easily understand what's in each one.

\emph{Example: For my images, I want to know date, sample ID, and image number for that sample on that date.}

\begin{enumerate}
\def\labelenumi{\arabic{enumi}.}
\tightlist
\item
\item
\item
\item
\item
\end{enumerate}

\textbf{3. Do you need to abbreviate any of the metadata or encode it?}

If any of the metadata from step 2 is described by lots of text, decide what shortened information to keep. If any of the metadata from step 2 has regular categories, standardize the categories and/or replace them with 2- or 3-letter codes; be sure to document these codes.

\emph{Example: Sample ID will use a code made up of: a 2-letter project abbreviation (project 1 = P1, project 2 = P2); a 3-letter species abbreviation (mouse = ``MUS'', fruit fly = ``DRS''); and 3-digit sample ID (assigned in my notebook).}

~

~

~

\textbf{4. What is the order for the metadata in the file name?}

Think about how you want to sort and search for your files to decide what metadata should appear at the beginning of the file name. If date is important, use ISO 8601-formatted dates (YYYYMMDD or YYYY-MM-DD) at the beginning of the file names so dates sort chronologically.

\emph{Example: My sample ID is most important so I will list it first, followed by date, then image number.}

\begin{enumerate}
\def\labelenumi{\arabic{enumi}.}
\tightlist
\item
\item
\item
\item
\item
\end{enumerate}

\textbf{5. What characters will you use to separate each piece of metadata in the file name?}

Many computer systems cannot handle spaces in file names. To make file names both computer- and human-readable, use dashes (-), underscores (\_), and/or capitalize the first letter of each word in the file names.

\emph{Example: I will use underscores to separate metadata and dashes between parts of my sample ID.}

~

~

~

\textbf{6. Will you need to track different versions of each file?}

You can track versions of a file by appending version information to end of the file name. Consider using a version number (e.g.~``v01'') or the version date (use ISO 8601 format: YYYYMMDD or YYYY-MM-DD).

\emph{Example: As each image goes through my analysis workflow, I will append the version type to the end of the file name (e.g.~``\_raw'', ``\_processed'', and ``\_composite'').}

~

~

~

\textbf{7. Write down your naming convention pattern.}

Make sure the convention only uses alphanumeric characters, dashes, and underscores. Ideally, file names will be 32 characters or less.

\emph{Example: My file naming convention is ``SA-MPL-EID\_YYYYMMDD\_\#\#\#\_status.tif'' Examples are ``P1-MUS-023\_20200229\_051\_raw.tif'' and ``P2-DRS-285\_20191031\_062\_composite.tif''.}

~

~

~

\textbf{8. Document this convention in a README.txt (or save this worksheet) and keep it with your files.}

~

~

~

\hypertarget{data-storage}{%
\chapter{Data Storage}\label{data-storage}}

\hypertarget{backup}{%
\section{Back Up the Data}\label{backup}}

\textbf{\emph{Description:}} **

\textbf{\emph{Instructions:}} **

\begin{center}\rule{0.5\linewidth}{0.5pt}\end{center}

\hypertarget{data-management}{%
\chapter{Data Management}\label{data-management}}

\hypertarget{living-dmp}{%
\section{Write a Living Data Management Plan}\label{living-dmp}}

\textbf{\emph{Description:}} **

\textbf{\emph{Instructions:}} **

\begin{center}\rule{0.5\linewidth}{0.5pt}\end{center}

\hypertarget{data-governance}{%
\section{Determine Data Stewardship}\label{data-governance}}

\textbf{\emph{Description:}} \emph{It is often helpful to be up front about requirements and permissions around research data. This exercise encourages you to discuss these issues with supervisors and peers to make sure that there are no misunderstandings about who has what rights to use, retain, and share data.}

\textbf{\emph{Instructions:}} \emph{Determine which research data should be discussed. Bring together the Principle Investigator, the researcher collecting the data, and anyone else who works with that data. As a group, answer the questions in the exercise, making sure that everyone agrees on the final decisions. Record the results of the discussion and save them with the project files.}

\textbf{\emph{Source:}} \emph{This exercise was adapted from the ``Project Close Out Checklist'' \citep{briney_project_2020}.}

\begin{center}\rule{0.5\linewidth}{0.5pt}\end{center}

\textbf{What data do these agreements apply to?}

~

~

~

\textbf{Are there security or intellectual property restrictions on the data and, if so, what are they?}

~

~

~

\textbf{Who will store the copy of record of the data and for how long?}

~

~

~

\textbf{Are there any requirement to share the data and, if so, what are they?}

~

~

~

\textbf{Who is allowed to reuse the data after the project ends? Are there any requirements for reuse?}

~

~

~

\textbf{Who keeps any physical research notebooks after the project ends?}

~

~

~

\hypertarget{data-sharing}{%
\chapter{Data Sharing}\label{data-sharing}}

\hypertarget{data-repository}{%
\section{Pick a Data Repository}\label{data-repository}}

\textbf{\emph{Description:}} **

\textbf{\emph{Instructions:}} **

\begin{center}\rule{0.5\linewidth}{0.5pt}\end{center}

\hypertarget{share-data}{%
\section{Share Data}\label{share-data}}

\textbf{\emph{Description:}} **

\textbf{\emph{Instructions:}} **

\begin{center}\rule{0.5\linewidth}{0.5pt}\end{center}

\hypertarget{project-wrap-up}{%
\chapter{Project Wrap Up}\label{project-wrap-up}}

\hypertarget{future-use}{%
\section{Prepare Data for Future Use}\label{future-use}}

\textbf{\emph{Description:}} \emph{The end of a project is a good time to prepare data for potential future reuse, as you still know the important details about the data to record and have access to any software used to create the data. This checklist exercise walks you through steps to gather your data into a central place and document the project. Working through the checklist results in project data being in one central location, well documented, and organized and formatted in a way to make future reuse easier.}

\textbf{\emph{Instructions:}} \emph{Gather all of the data from a project and work through the checklist to organize and document the data for future reuse. This exercise refers to several other exercises in the Workbook that should be completed during this process, if they have not been already}

\textbf{\emph{Source:}} \emph{This exercise was adapted from the ``Project Close Out Checklist'' \citep{briney_project_2020}.}

\begin{center}\rule{0.5\linewidth}{0.5pt}\end{center}

\textbf{Prepare Data}

□ Move all data into one central project folder; this folder may have sub-folders and should be organized however makes sense for your data.

□ As necessary, copy data into more open/common file formats; see \protect\hyperlink{file-type}{Exercise 7.2: Convert Data File Types}.

\textbf{Back Up Your Research Notes}

□ If your notes are electronic, save a copy in the project folder

□ If your notes are physical, scan them and save a copy in the project folder.

\textbf{Create a Project Archive Folder}

□ Work through \protect\hyperlink{archive-folder}{Exercise 7.3: Create an Archive Folder}.

□ Put the Archive folder in the project folder.

\textbf{Create a Project-Level README File}

□ Work through \protect\hyperlink{readme-txt}{Exercise 2.2: Project-Level README.txt}.

□ Store a copy of the README file with the data.

\textbf{Save Files in a Stable Location}

□ Save the project folder on a storage system that you will have access to for the next several years.

\hypertarget{file-type}{%
\section{Convert Data File Types}\label{file-type}}

\textbf{\emph{Description:}} \emph{Data is often stored in a file type that can only be opened by specific, costly software -- this is referred to as a ``proprietary file type.'' You can tell that you have data in a proprietary file type if you lose access to the data when you lose access to the software. When data is in a propreitary file type, it's always a good idea to copy the data into a more common, open file type as a backup; you may lose a bit of functionality, but it's better to have a backup than to not have your data at all! This exercise works through identifying possible alternative file types for the data's proprietary file type before instructing you to make a copy of the data in the new file type.}

\textbf{\emph{Instructions:}} \emph{For any data in a proprietary file type, identify the data and answer the following questions. Once you have picked a more open, common file type, make a copy of the data in that file type but do not delete the original data.}

\begin{center}\rule{0.5\linewidth}{0.5pt}\end{center}

\textbf{Is your data stored in a proprietary file type? What file type and how does this limit future data reuse?}

~

~

~

\textbf{Is it possible to convert your data to other file types? If so, list the possible types:}

~

~

~

\textbf{Which of the possible file types are in common use? Which of the possible file types can be opened by multiple software programs?}

~

~

~

\textbf{Of the possible options above, do you have a preference for a specific file type?}

~

~

~

\textbf{Pick one of the more open or common file types and copy your important data files into that file type. Do not delete the original files.}

~

~

~

\hypertarget{archive-folder}{%
\section{Create an Archive Folder}\label{archive-folder}}

\textbf{\emph{Description:}} \emph{To save your future-self time spent digging through all of your research files, set aside the most important files into a separate ``Archive'' folder. Do this at the end of the project while you still remember which files are important and where they are located. The Archive folder should only contain a small subset of the most important documents that are likely to be reused; you may still need to go through all of your files but, in the majority of instances, you will save time by easily finding what you need in the Archive folder.}

\textbf{\emph{Instructions:}} \emph{This exercise consists of a checklist of the key documents that are likely to be most useful in a research project archive. Create a separate folder within the larger project folder (or in a highly visible place within the storage system) labelled ``Archive''. Copy -- do not move -- the files on this checklist into the Archive folder. Add copies other important research documents, as needed. Remember, the Archive folder does not need to be comprehensive, so focus on the subset of files that are most likely to be reused or referenced in the future.}

\textbf{\emph{Source:}} \emph{This exercise was adapted from the ``Project Close Out Checklist'' \citep{briney_project_2020}.}

\begin{center}\rule{0.5\linewidth}{0.5pt}\end{center}

\textbf{Project Documentation}

□ README file of project information

\textbf{Data Snapshots}

□ Raw data

□ Key data analyses

□ Final data

\textbf{Code}

□ Analysis code

□ Record software version, as appropriate

\textbf{Other Research Documents}

□ Protocols

□ Survey instruments

\textbf{Research Notes}

□ Scan of research notebook

□ Digital notes

\textbf{Images}

□ Flat files of figures (e.g.~.JPG or .TIFF)

□ Editable image files (e.g.~Photoshop)

\textbf{Publications}

□ Published article in .PDF format

□ Final version of the article in editable document format (e.g.~.DOCX)

□ Posters

\textbf{Administrative Documents}

□ Grant proposals

□ Grant progress reports and final report

\hypertarget{separation}{%
\section{Separate from the Institution}\label{separation}}

\textbf{\emph{Description:}} **

\textbf{\emph{Instructions:}} **

\textbf{\emph{Source:}} \emph{This exercise was adapted from the ``Data Departure Checklist'' \citep{goben_data_2023}.}

\begin{center}\rule{0.5\linewidth}{0.5pt}\end{center}

  \bibliography{references.bib}

\end{document}
